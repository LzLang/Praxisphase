\documentclass[margin=0mm, innermargin=3cm, blockverticalspace=15mm, colspace=15mm, subcolspace=8mm, a0paper, titleinnsersep = 0pt]{tikzposter}
\usepackage{graphicx} 
\usepackage[utf8]{inputenc}
\usepackage{xcolor}
\usepackage{tikz}
\usetikzlibrary{shapes.geometric, arrows.meta}
\usepackage[ngerman]{babel}
\usepackage{multicol}
\usepackage{wrapfig}
\tikzposterlatexaffectionproofoff %Tikzposter Wasserzeichen ausblenden
\usetikzlibrary{babel}
\setlength{\columnsep}{2cm}

%Abbildungsreferenzen Deutsch
\addto\captionsngerman{
  \renewcommand{\figurename}{Abb.}
  \renewcommand{\tablename}{Tab.}
}

%Titel
\title{\parbox{0.6\linewidth}{\centering Robustness of machine learning against Batch effects in RNA-seq data}}

%Farbpalette in TH-Farbe von Originalvorlage
\definecolorpalette{thpalette} {
    \definecolor{thblue}{RGB}{0,91,153}
}

%Hintergrund: Rahmen etc.
\definebackgroundstyle{frame}{
    %Aussenrahmen
    \draw[inner sep=0pt, line width=30pt, color=thblue, fill=white]
    (-40,-57) rectangle (40, 57);
    %Linien box Praxisphase von x,y
    \draw[line width=15pt, color=thblue] (-40,45) -- (40,45);
    \draw[line width=15pt, color=thblue] (-40,39) -- (40,39);
    %Linien box Durchgefuehrt bei: Betrieb, Ort
    %\draw[line width=15pt, color=thblue] (-40,-45) -- (40,-45);
    \draw[line width=15pt, color=thblue] (-40,-51) -- (40,-51);
}

%Titel Style
\definetitlestyle{thtitle}{
    width=\paperwidth, roundedcorners=0, linewidth=0pt, innersep=1.5cm,
    titletotopverticalspace=5cm, titletoblockverticalspace=6cm,
    titlegraphictotitledistance=1pt, titletextscale=1
}{}

%Block Style
\defineblockstyle{thblocks}{
titlewidthscale=0.9, bodywidthscale=1,titleleft,
titleoffsetx=0pt, titleoffsety=3mm, bodyoffsetx=0mm, bodyoffsety=19mm,
bodyverticalshift=18mm, roundedcorners=5, linewidth=2pt,
titleinnersep=6mm, bodyinnersep=1cm
}{
\draw[color=thblue, fill=blockbodybgcolor] (blockbody.south west)
rectangle (blockbody.north east);
\ifBlockHasTitle
\draw[color=thblue, fill=thblue] (blocktitle.south west)
rectangle (blocktitle.north east);
\fi
}

%TH layout definieren
\definelayouttheme{th}{
    \usebackgroundstyle{frame}
    \usecolorpalette{thpalette}
    \usetitlestyle{thtitle}
    \useblockstyle{thblocks}
}

\usetheme{th} 
\begin{document}

%Logos einfuegen
\node (thlogo) at (32, 51) {\includegraphics[width=10cm]{bingen.pdf}};
\node (betrieb) at (-32, 51) {\includegraphics[width=10cm]{mainz.png}};

%Praxisphase von Vorname Name
\node [above right,
    outer sep=0pt,
    fill opacity=0,
    line width=0mm,
    text opacity=1,
    minimum width=80cm,
    minimum height=6cm,
    align=center,font=\Huge,
    draw=none,fill=white] at ([shift={(0.5*\pgflinewidth,0.5*\pgflinewidth)}]-40,39) {Practical phase of Laszlo Lang\\Supervised by Dr. Maximilian Sprang – The Mayer Lab Mainz};

%Durchgefuehrt bei Firma, Ort
%\node [above right,
%    outer sep=0pt,
%    fill opacity=0,
%    line width=0mm,
%    text opacity=1,
%    minimum width=80cm,
%    minimum height=6cm,
%    align=center,font=\Huge,
%    draw=none,fill=white] at ([shift={(0.5*\pgflinewidth,0.5*\pgflinewidth)}]-40,-51) {Durchgeführt bei: %Betrieb, Ort};

%Studiengang Angewandte Bioinformatik
\node [above right,
    outer sep=0pt,
    fill opacity=0,
    line width=0mm,
    text opacity=1,
    minimum width=80cm,
    minimum height=6cm,
    align=center,font=\Huge,
    draw=none,fill=white] at ([shift={(0.5*\pgflinewidth,0.5*\pgflinewidth)}]-40,-57) {Bachelor's Degree Program „Angewandte Bioinformatik“};

\maketitle[]

